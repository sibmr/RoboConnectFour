\documentclass[11pt,pdftex,a4paper]{article}
\usepackage[utf8]{inputenc} 
\usepackage[english]{babel}
\usepackage{amsmath}
\usepackage{algorithm}
\usepackage[noend]{algpseudocode}
\usepackage{amssymb}
\usepackage{bbm}
\usepackage{amsmath,amssymb,amsthm,mathrsfs,amsfonts,dsfont} 
\newcommand{\bbN}{\mathbbm{N}}
\newcommand{\bbR}{\mathbbm{R}}
\newcommand{\bbZ}{\mathbbm{Z}}
\newcommand{\bbI}{\mathbbm{I}}
\usepackage[pdftex]{graphicx}
\usepackage{listings}
\usepackage{paralist}
\usepackage{booktabs}
\usepackage{wrapfig}
\usepackage{tikz}
\usepackage{verbatim}


\pagestyle{empty}
%\lstset{language=Python,basicstyle=\footnotesize}
\usepackage{tabularx}
\usepackage{multirow}
\usepackage[top = 2cm, left = 2cm, right = 2cm, bottom = 2cm]{geometry}
\usepackage{hyperref}
\usepackage{xcolor, soul}
\usepackage{pdfpages}

\usepackage{fancyhdr}
\usepackage{fancyvrb}



% redefine \VerbatimInput
\RecustomVerbatimCommand{\VerbatimInput}{VerbatimInput}%
{fontsize=\footnotesize,
	%
	frame=lines,  % top and bottom rule only
	framesep=2em, % separation between frame and text
	rulecolor=\color{Gray},
	%
	label=\fbox{\color{Black}data.txt},
	labelposition=topline,
	%
	commandchars=\|\(\), % escape character and argument delimiters for
	% commands within the verbatim
	commentchar=*        % comment character
}





%---------------------------------------------------------------------------

\definecolor{RedCode}{RGB}{173,16,8}
\definecolor{GreenCode}{RGB}{86,168,5}
\definecolor{StringCode}{RGB}{160,32,240}
\definecolor{Black}{RGB}{0,0,0}
\definecolor{BlueCode}{RGB}{20,68,123}






%---------------------------------------------------------------------------


\makeatletter
\def\BState{\State\hskip-\ALG@thistlm}
\makeatother
% Überschreibt enumerate Befehl, sodass 1. Ebene Items mit
% (a), (b), etc. nummeriert werden.
%\renewcommand{\theenumi}{(\alph{enumi})}
\renewcommand{\labelenumi}{\text{\theenumi}}


\newcommand{\image}[1]{


\begin{center}
	\includegraphics[width = 1.0\textwidth]{img/#1}
\end{center}
}

\definecolor{exco}{rgb}{0,.4,.7}

% Counter für das Blatt und die Aufgabennummer.
% Ersetze die Nummer des Übungsblattes und die Nummer der Aufgabe
% den Anforderungen entsprechend.
% Beachte:
% \setcounter{countername}{number}: Legt den Wert des Counters fest
% \stepcounter{countername}: Erhöht den Wert des Counters um 1.
\newcounter{sheetnumber}
\newcounter{exnum}


 \renewcommand{\title}
{\begin{center}
		\begin{Huge}
            \textbf{Practical-course Robotics (SS20)}\\[0.3cm]
			 \textbf{Project proposal} \\[0.3cm]
		\end{Huge}
		
		\begin{tabular}{rl}
			Simon Bihlmaier	 	 & 3239454 \\
			Julian Obst 		 & 3228106 \\
			Thomas Monninger	 & 3470145 \\
		\end{tabular}
		
	\end{center}
	
}

\lhead{Simon Bihlmaier $\quad$ 3239454}
\chead{Julian Obst 	$\quad$ 3228106}
\rhead{Thomas Monninger $\quad$ 3470145}


\usepackage{multicol}
\pagestyle{fancy}	

\newcommand{\marker}[1]{\underline{\textcolor {black} {#1}}}

\newcommand{\ex}[1]{\section*{\theexnum $\quad$ #1 \stepcounter{exnum}}}

\newcommand{\nitem}[2]{\item [\texttt{#1}:] \hfill  \\ #2}

%Namen

\begin{document}
	
	
	%%Title
	\setcounter{exnum}{1}
	\setcounter{sheetnumber}{0}
	\thispagestyle{empty}
	\title

    \ex{Objective}
	The objective of this project is to realize the game ``Connect Four''. 
	The game consists of a play grid placed vertically. 
	The grid is opened at the top so that discs can be inserted. 
	For each column the coins stack up. 
	``Connect Four'' is a game between two players with differently coloured discs. 
	During the game the players alternately insert one of their discs in a column of their choice. 
	The goal of the game is to end up with 4 connected discs of the same color. 
	The disc line can be directed either vertically, horizontally or diagonally.
	
	To cover this objective, the setting of this game will be a simulation world. 
	This simulation world essentially contains a ``Connect Four'' grid and provides objects for the players to manipulate. 
	As a minimal requirement, the objects under manipulation will be balls. 
	This eases the grasping process, the insertion into the grid and avoids potential issues with simulating physics. 
	As an optional task, the simulation world will provide actual disc-shaped cylinders. 
	This will drastically increase complexity of the robot manipulation and shall therefore be considered as optional. 
	
	The robot arm will be actor in this setting and execute actions in the simulation world.
	In a minimal scope, one robot arm will act as player and realize all actions required to play the game ``Connect Four''. 
	This includes grasping of objects and inserting them into the grid of ``Connect Four''. 
	While executing actions, the actor shall find a reasonable path within the simulation environment. 
	Optionally, the behavior of the robot arm should be adaptive and account for fail-cases like misgrasping. 
	Furthermore, the described behavior could be realized for the second arm as well. 
	Therefore, the second robot arm  can act as an opponent in order to realize an interactive game mode.
	
	Third part of the objective will provide a game logic. 
	The logic defines how the agent acts given a specific game state.
	The first and most straight-forward logic will not follow a specific strategy. 
	This already yields interactive game play. 
	Optionally, the game strategy of the robot can be intelligent.
	A simple AI could be used to solve the game ``Connect Four''.
	 
	A fully optional part of the objective will be perception. 
	This is optional in alignment with the course project requirements. 
	One objective could be to perceive the occupancy matrix of the game grid.
	Additionally, the position of the objects on the table could be localized in order to enable grasping.
	
	The final contribution to the objective shall include some degree of human - robot interaction.
	As minimal requirement, the interaction with the human takes place via console input and simulation output. 
	In particular, for each turn the human user shall enter a column so that the robot can execute his or her move. 
	Optionally, webcam input or a graphical user interface could be used to make the user experience more immersive. 
	If optionally different game modes like human vs. robot or robot vs. robot are implemented the user interface shall provide a selection of game mode at start.

   
	\ex{Work Plan}
	In the following we will present our workpackages.
	These are further described by their main components.
	We made an estimate to what we think will be the time to complete a certain package.
	In general, all team members work on each workpackage.
	Personal preference lets each team member decide his priority on the pending workpackages.
	The packages are given in the order of their priority.

	\section*{Methodology}
	All high-level functionality is going to be implemented in an expandable and modularized design.
	This means that generic functions will be implemented in a library
	One example for modularized implementation is the robot arm manipulation. 
	This shall be programmed in a generic way such that the functions are applicable for both robot arms with just one implementation.
	In regard of the AIs a modular approach enables the use of equal or different strategies in the optional robot vs. robot scenario.
	Therefore, the quality of strategies can directly be compared.

	\section*{Workpackages}
	The assignment of workpackages to persons is not final. 
	Given the fact that the effort of each workpackage is not fully determined task assignment may change during progress.\\
	Mandatory tasks will be completed as minimial scope and are denoted with `M'.\\
	Optional tasks represent additional effort exceeding the base features and are denoted with `O'.
	\\

	\noindent
	\textbf{Build simulation world, create g file} (10h) - Simon, Julian\\
	The simulation world shall provide a setup of the game ``Connect Four''.
	\begin{itemize}
	\item M - Define 3D grid of the game ``Connect Four'' (different shapes of objects to put in)
	\item M - Change Robot placement according to the game setting
	\item O - Define camera placement to cover all required detections
	\item M - Create game objects (balls or disc) statically
	\item O - Spawn game objects (balls or discs) dynamically in a random fashion on or above the table	
	\end{itemize}
	
	\noindent
	\textbf{Framework} (10h) - Thomas\\
	The framework covers all architectural components of the software.
	\begin{itemize}
	\item M - Define classes towards modularized approach
	\item M - Handle simulation and configuration space
	\item M - Manage simulation steps in accordance with manipulation constraints
	\item M - Define interfaces to game logic, perception and user
	\end{itemize}
	
	\noindent
	\textbf{State machine} (16h) - Simon, Thomas\\
	The state machine shall define the behavior of a robot arm. This covers its states and transitions.
	\begin{itemize}
	\item M - Define robot-internal states
	\item M - Derive state transition framework
	\item M - Implement state machines to cover different behaviours 
	\end{itemize}
	
	\noindent
	\textbf{Manipulation} (30h) - Simon, Julian, Thomas\\
	Manipulation defines the movements and actions of the robot.
	\begin{itemize}
	\item M - Motion: Move to position, align orientation, ...
	\item M - Implement gripper controls (Grasping and Dropping)
	\item M - Insert object (ball or disc) into ``Connect Four'' grid
	\item M - Collision Handling
	\item O - Perturbation of objects (balls or discs) 
	\item O -Reachability of objects
	\item M - Find appropriate threshold values for state transitions
	\end{itemize}
	
	\noindent
	\textbf{Game agent} (10h) - Thomas, Julian, Simon\\
	The game agent yields an action for a given game state.
	\begin{itemize}
	\item M - Game mode 1:  Robot vs. Human
	\item O - Game mode 2:  Robot vs. Robot
	\item M - Define game states and transitions
	\item M - Random strategy
	\item O - Optimal strategy (AI)
	\end{itemize}
	
	\noindent
	\textbf{Perception} (5-20h depending on focus) - Julian\\
	The perception provides methods to detect the occupancy of the game grid and to locate objects on the table.
	\begin{itemize}
	\item O - Implement generic computer vision methods
		\begin{itemize}
		\item Contour detection and segmentation
		\item Filtering
		\item Object identification
		\item Object tracking	
		\end{itemize}
	\item O - Detect ``Connect Four'' grid occupation
	\item O - Detect, classify and track objects (balls or discs) on table
	\end{itemize}
	
	\noindent
	\textbf{Human UI/UX} (5-15h depending on implementation) - Julian\\
	Basic human - robot interaction for the game ``Connect Four''.
	\begin{itemize}
	\item M - Develop user interface for selection of game mode
	\item M - Handle input from console
	\item O - Handle input from alternative sources (webcam, ...)	
	\item O - Develop graphical user interface to facilitate inputs
	\item M - Provide simulation rendering to user
	\item O - Display game score to user
	\end{itemize}
	
	\begin{figure}[H]
	\image{Connect-4-concept}
	\caption{Simulation world with robot and schematic ``Connect Four'' grid.}
	\end{figure}

\end{document}

